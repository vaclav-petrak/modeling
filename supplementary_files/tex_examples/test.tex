\documentclass[border=1cm]{standalone}

\usepackage{mathtools}
\usepackage{mathdots}

\usepackage{pgf}
\usepackage{tikz}
\usetikzlibrary{positioning}
\usetikzlibrary{decorations.pathreplacing}
\usetikzlibrary{shapes.geometric}
\usetikzlibrary{automata}
\usetikzlibrary{arrows}
\tikzset{pics/.cd,
myinput/.style args={#1#2#3#4#5#6#7}{
code={% #1=color, #2=x1, #3=y1, #4=edge style #5=x2,#6=y2, #7=label
\draw[color=#1]  (input)  +(#2,#3) edge[#4] +(#5,#6);
\draw[](input)  +(-9pt,-12pt) -- +(-9pt,12pt);
\draw[rotate=90]  (input)  +(-9pt,-12pt) --node[below=0.3cm,midway](){#7} +(-9pt,12pt);}}
 }
\begin{document}
\tikzstyle{colorful} = [fill=gray!10]
\tikzstyle{block} = [draw, colorful, rectangle, minimum height=3em, minimum width=3em]
\tikzstyle{ramp}=[block,pin={below:Ramp}]
\tikzstyle{sum} = [draw, colorful, circle, node distance=1cm]
\tikzstyle{gain} = [draw, colorful, regular polygon, regular polygon sides=3, shape border rotate=30]
\tikzstyle{integrator} = [block, right=of gain, pin={below:Integrator}]%, node contents = $\frac{1}{s}$]
\tikzstyle{input} = [coordinate]
\tikzstyle{output} = [coordinate]
\tikzstyle{node} = [coordinate]
\tikzset{every pin/.style={pin distance = 1mm}}
\tikzset{every pin edge/.style={draw=none}}

\begin{tikzpicture}[auto, node distance=1cm,>=latex']
    \node [block] at (0,0) (input) {};
    \node [block, right=of input] (prod) {$\times$};
    \node [gain, right=of prod] (gain) {$-1$};
    \node [integrator] (system) {$\frac{1}{s}$};
    \node [output, right=of system] (output) {};
    \draw [draw,->] (gain) -- node [name=dy] {$\dot{y}$} (system);
    \draw [draw,->] (system) -- node [name=y] {$y$} (output);
    \draw [draw,->] (input) -- node {$t$} (prod);
    \draw [draw,->] (prod) -- node {} (gain);

    \node [node, below=of y] (node1) {};
    \node [node, below=of prod] (node2) {};
    \draw [->] (y) |- (node1) |- (node2) -- (prod);

       % Initial value for integrator
    \node [above=0.5cm of system] (init) {$x_0$};
    \draw [->] (init) -- (system);


    % ramp function
    \pic {myinput={red}{-9pt}{-9pt}{out=45,in=225}{12pt}{12pt}{}};
    % step function
    \pic {myinput={blue}{-9pt}{5pt}{out=0,in=180}{12pt}{5pt}{}};
    % parabolic function
    \pic  {myinput={cyan}{-9pt}{-9pt}{out=0,in=-110}{12pt}{12pt}{}};
\end{tikzpicture}
\end{document}