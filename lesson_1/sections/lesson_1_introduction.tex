\section{Introduction}
\subsection{Course information}
\begin{frame}{Contacts}
    \textbf{Lecturer:} Václav B. Petrák  

    \textbf{Email:} vaclav.petrak@fbmi.cvut.cz  

    \textbf{Phone/WhatsApp:} +420 725 878 390

    \vspace{1em}
    \textbf{Consultations:} 
    \begin{itemize}
        \item In person: after individual request
        \item In person: on Friday 10\textsuperscript{th} of May 2024 for project-related problems
        \item By email any time
    \end{itemize}

\end{frame}

\begin{frame}{Absence Policy}


\begin{itemize}
    \item \textbf{Advance Notice Required:} All absences must be excused in advance whenever possible.

    \item \textbf{Communication Procedure:}  Contact me  as soon as possible by email with date of absence a reason for the absence.

    \item \textbf{Responsibility for Missed Work:} You will be are responsible for making up missed assignments, by following the content of course materials.  
\end{itemize}
 

\end{frame}


\begin{frame}{Lectures}
\Large
\begin{itemize}
    \item We will have 5 topics in modeling and simulation, each for one day.
    \item The last lecture will be focused on presentation of your work. 
\end{itemize}
\end{frame}

\begin{frame}{Classification}
\begin{center}
    \Large
    There is no exam.
    
    \vspace{1em}
    %\pause
    Your grade will be based on your independent work.
    %\pause

    \vspace{1em}
    \color{primary}
    \textbf{1. Journal Club Presentation}
    
    %\pause
    \vspace{1em}
    \color{secondary}
    \textbf{2. Capstone Project} 
\end{center}
\end{frame}

\begin{frame}{Journal Club is a discussion of research papers.}
   A journal club is a group who meet to critically evaluate  articles in the academic literature. Journal clubs:
     \begin{itemize}
        \item Encourages critical thinking.
        \item Helps in understanding the studied concepts and methods.
        \item Helps keep up with the latest research.
        \item Improves communication and presentation skills.
        \item Presentation simulates experience of a scientist
    \end{itemize}
\vspace{1em}
\textbf{The main purpose in this course is to explore how methods we learned are applied in current research}
    
\end{frame}
\begin{frame}{Your Task for Journal Club}
    \begin{itemize}
        \item You will be assigned a research paper related to the course topic.
        \item Your task is to read the paper and prepare a 15-minute presentation, followed by a 5-minute discussion, to explain the paper to your fellow students.
        \item The task will be graded by the lecturer and your classmates. 
    \end{itemize}
\end{frame}

\begin{frame}
\frametitle{The Presentation Will Cover}
    \begin{itemize}
        \item \textbf{High-level Overview:} Provide a background of the research topic and the main objectives of the paper.
        \item \textbf{Methods:} Give overview of the methodology used in the paper.
        \item \textbf{Results:} Highlight the key results and  conclusions of the paper.
        \item \textbf{Implications and Applications:} Discuss the practical implications of the research and its applications in real-world situations.
        \item \textbf{Limitations:} Discuss any limitations of the paper.
        \item \textbf{Discussion:} Engage in a 5-minute discussion with other students, encouraging questions and opinions.
    \end{itemize}
\end{frame}

\begin{frame}[t]{Grading Criteria for Journal Club Presentation}
\begin{itemize}
\only<1-2>{
    \item \textbf{Presentation (35\%)}
    \begin{itemize}
        \item \textbf{Design:} The presentation is well-organized and easy to follow. The slides are well-integrated into the flow of the presentation.
        \item \textbf{Timing:} The presenter adhered to the 15-minute presentation format.
        \item \textbf{Delivery:} Clear and confident speech with appropriate pacing.
    \end{itemize}
    }
    
\only<2>{   
   \item \textbf{Content of the presentation (35\%)}
    \begin{itemize}
       \item \textbf{Introduction:} The introduction provides a high-level overview of the topic. The presenter explains the research motivation (why) and the research question (what).
       \item \textbf{Methods:} The presenter explained the experimental approach used by the authors. Unfamiliar methods are explained without going into excessive detail.
       \item \textbf{Results:} The presenter showed and explained the key results from the paper.
    \end{itemize}
}

\only<3-4>{
    \item \textbf{Critical Thinking (15\%):} The presenter provides a critical evaluation of the paper that may consist of:
    \begin{itemize}
        \item Critique of methods or experimental design.
        \item Ideas for improvement and further research.
        \item Discussion of how the results relate to the research question.
        \item Significance of the results.
    \end{itemize}
}
\only<4>{
    \item \textbf{Questions  (15\%):} The presenter is able to respond convincingly to audience questions.
}
\end{itemize}
\end{frame}


\begin{frame}{Capstone project}
\begin{columns}
    \begin{column}{0.55\textwidth}
        \begin{itemize}
            \item You will work on independent project that builds on or extends course content.  
            \item Output of your work will be a poster, presented on Friday 17\textsuperscript{th} of May 2024 --- take maximum effort to be present
            \item You will also prepare a written paper due two weeks before the end of the exam period. 
            \item The projects will be assigned on 19\textsuperscript{th} of April 2024.
        \end{itemize}
    \end{column}
    %\pause
    \begin{column}{0.45\textwidth}
    \footnotesize
    \begin{center}
        \includegraphics[scale = 0.75]{lesson_1/images/capstone.png} 
        \end{center}
    \vspace{1em}
       \textit{ A  \textbf{capstone} (or keystone) is the wedge-shaped stone at the apex of an arch. \textbf{It is the final piece placed during construction and locks all the stones into position}, allowing the arch to bear weigh.}
 
    \end{column}
\end{columns}
    
\end{frame}

\begin{frame}{Avoid using AI to solve entire exercises}
    \begin{itemize}
        \item Consider a math problem: 
        \begin{itemize}
            \item \textit{Given that two trains are traveling on parallel tracks in opposite directions with speeds of 56 km/h and 52 km/h, respectively, and lengths of 136 m and 242 m, respectively, after what duration will they cross each other?}
        \end{itemize}
        %\pause
        \item The primary goal is \textbf{not} to find the answer but to practice your reasoning and calculation skills.
           %\pause
        \item Similarly,  the aim of modeling and simulation exercises is to learn and practice new skills.
           %\pause
        \item Avoid using generative AI like ChatGPT or Gemini to solve the \textbf{entire problem} for you.      
        \begin{itemize}
            \item It spoils learning process for you
            \item It may disrupt pacing of the lesson. 
        \end{itemize}
    \end{itemize}
\end{frame}


\begin{frame}{Using AI During Lessons} % More direct title
\Large 
During lessons, you may use AI to:

\begin{itemize}
    \item Ask clarifying questions. 
    \item Debug your code: after you've made your own attempts. 
    \item Get ideas for further work. 
\end{itemize}

\vspace{0.5cm}  % Increased spacing

\begin{center}
\textbf{Let's discuss what is a fair use of generative AI in the capstone project} 
\end{center}
\end{frame}

\subsection{Introduction to Modeling}
\begin{frame}[t]{Model is a representation of a \textbf{system}}
\small
 \begin{columns}
 \begin{column}{0.65\textwidth}
     \begin{itemize}
    \item    A model is a representation of a \textbf{system} that enables us investigate its properties and \textbf{predict} future outcomes.
    \begin{itemize}
           \item    A \textbf{system} is  a set of components  that interact with each other within a boundary to function as a whole: Solar system, rabbit and fox populations, a microscope
           
       \end{itemize}
    \item In modeling we aim to identify components, relationships and behavior to predict system dynamics.
       \only<2->{
        \item Modeling always requires simplification (abstraction)
       }
      \only<3>{
         \item  \textbf{A mathematical model uses mathematical equations to describe a system.}}
    \end{itemize}
 \end{column}

 \begin{column}{0.35\textwidth}
 \color{primary}

\only<1>{
\begin{center}
   \includegraphics[scale = 0.15]{lesson_1/images/dummy.png}  
\end{center}}

 \only<2>{
 Example of a model: 
SIR (susceptible, infectious, and recovered) model for dynamics of infectious diseases
\vspace{1em}

\includegraphics[scale = 0.12]{lesson_1/images/sir.png}
 }
  \only<3>{
 SIR  model equations:
\begin{align*}
    \frac{dS}{dt} &= -\beta S I \\
    \\
    \frac{dI}{dt} &= \beta S I - \gamma I \\
    \\
    \frac{dR}{dt} &= \gamma I
  \end{align*}
 }
 \end{column}
 \end{columns} 
\end{frame}

{
\setbeamercolor{background canvas}{bg=black}
\setbeamercolor{normal text}{fg=white}
\begin{frame}
\begin{center}
\large
    \color{white} % This ensures that the text is white
    All models are approximations. Assumptions, whether implied or clearly stated, are never exactly true.
    %\pause
    
    \vspace{1em}
    \textbf{All models are wrong, but some models are useful.} 

    %\pause
    
    \vspace{1em}
    So the question you need to ask is not \textit{Is the model true?} 
    
        %\pause
    \vspace{1em}   
    It never is.
        %\pause

    \vspace{1em}
    But \textit{Is the model good enough for this particular application?}
  \vfill % This ensures vertical space is added before your text at the bottom
\hfill --- George E. P. Box % This will align "George E. P. Box" to the right
\end{center}


\end{frame}
}
