\documentclass{beamer}
\usepackage{graphicx}
\usepackage{amsmath}
\usepackage{amssymb}
\usepackage{booktabs} % For better tables, if needed
\usepackage{multicol} % For columns
\usepackage{lmodern} % Or any other font package you prefer
\usepackage{xcolor} % For colors

% Define your colors (matching your existing setup if possible)
\definecolor{primary}{rgb}{0.2, 0.2, 0.7} % Example primary color
\definecolor{secondary}{rgb}{0.7, 0.2, 0.2} % Example secondary color

\usetheme{default} % Or your preferred theme

% You can put your \title, \author, \institute, \date here
% \title{Modeling and Simulation}
% \author{Václav B. Petrák}
% \institute{FBMI CVUT}
% \date{\today}

\begin{document}

% \begin{frame}
% \titlepage
% \end{frame}

% Assuming this follows your "Classification" slide which lists:
% 1. Journal Club Presentation
% 2. Capstone Project
% 3. Code Review

\section{Capstone Project Grading}

\begin{frame}[t]
\frametitle{Grading Criteria for Capstone Project}
The Capstone Project grade is based on the implementation, analysis, presentation, and a written abstract.
\vspace{1em}

Assumed contribution to final grade: \textbf{40\%} (adjust as needed)
\vspace{1em}

\begin{itemize}
    \item The project involves developing a model, implementing core and additional features, analyzing results, and presenting findings.
    \item Deliverables: Presentation and a scientific abstract.
\end{itemize}
\end{frame}

%-------------------------------------------------------------------------------

\begin{frame}[t]
\frametitle{Capstone: Project Implementation and Technical Merit (40\% of Capstone Grade)}
\begin{itemize}
    \item \textbf{Core Model Implementation (15\%)}
    \begin{itemize}
        \item Accuracy and correctness of the fundamental model.
        \item Demonstrates understanding of core concepts.
    \end{itemize}
    \vspace{0.5em}
    \item \textbf{Additional Features Implementation (15\%)}
    \begin{itemize}
        \item Successful and meaningful implementation of $\geq 2$ additional features.
        \item Complexity, creativity, and integration of features.
    \end{itemize}
    \vspace{0.5em}
    \item \textbf{Code Quality (Preliminary) (10\%)}
    \begin{itemize}
        \item Code is reasonably organized and understandable.
        \item Appropriate use of programming constructs.
    \end{itemize}
\end{itemize}
\end{frame}

%-------------------------------------------------------------------------------

\begin{frame}[t]
\frametitle{Capstone: Analysis, Interpretation and Critical Thinking (30\% of Capstone Grade)}
\begin{itemize}
    \item \textbf{Results Demonstration and Visualization (10\%)}
    \begin{itemize}
        \item Clear and effective presentation of simulation results (graphs, animations, etc.).
        \item Results address project objectives and show model behavior.
    \end{itemize}
    \vspace{0.5em}
    \item \textbf{Analysis of Parameters and Emergent Patterns (10\%)}
    \begin{itemize}
        \item Insightful discussion on how parameters/initial conditions affect outcomes.
        \item Identification and explanation of emergent behaviors.
    \end{itemize}
    \vspace{0.5em}
    \item \textbf{Discussion of Limitations and Real-World Comparison (10\%)}
    \begin{itemize}
        \item Thoughtful consideration of model limitations and assumptions.
        \item Comparison to real-world data, phenomena, or existing research.
    \end{itemize}
\end{itemize}
\end{frame}

%-------------------------------------------------------------------------------

\begin{frame}[t]
\frametitle{Capstone: Presentation and Communication (20\% of Capstone Grade)}
\begin{itemize}
    \item \textbf{Structure and Clarity (10\%)}
    \begin{itemize}
        \item Well-organized, logical, and easy-to-follow presentation.
        \item Clear explanation of goals, methods, features, and results.
        \item Effective use of visual aids.
    \end{itemize}
    \vspace{0.5em}
    \item \textbf{Delivery and Engagement (5\%)}
    \begin{itemize}
        \item Clear, confident speech with appropriate pacing.
        \item Engages the audience effectively.
    \end{itemize}
    \vspace{0.5em}
    \item \textbf{Adherence to Time (5\%)}
    \begin{itemize}
        \item Adherence to allocated presentation time (e.g., 20-25 min + Q \& A).
    \end{itemize}
\end{itemize}
\end{frame}

%-------------------------------------------------------------------------------

\begin{frame}[t]
\frametitle{Capstone: Abstract (10\% of Capstone Grade)}
\begin{itemize}
    \item \textbf{Content and Clarity (5\%)}
    \begin{itemize}
        \item Abstract clearly and concisely summarizes the project's purpose, methods, key findings, and significance.
    \end{itemize}
    \vspace{0.5em}
    \item \textbf{Accuracy and Completeness (5\%)}
    \begin{itemize}
        \item Accurately reflects the work done and main conclusions.
        \item Includes all necessary components of a scientific abstract.
    \end{itemize}
\end{itemize}
\end{frame}

%-------------------------------------------------------------------------------
\section{Code Review Grading}

\begin{frame}[t]
\frametitle{Grading Criteria for Code Review}
This component assesses the quality, design, and documentation of the Capstone Project code.
\vspace{1em}

Assumed contribution to final grade: \textbf{30\%} (adjust as needed)
\vspace{1em}

\begin{itemize}
    \item Focuses on the submitted codebase for the Capstone Project.
    \item May involve a code walk-through or defense session.
\end{itemize}
\end{frame}

%-------------------------------------------------------------------------------

\begin{frame}[t]
\frametitle{Code Review: Code Quality and Design (50\% of Code Review Grade)}
\begin{itemize}
\only{
    \item \textbf{Readability and Maintainability (15\%)}
    \begin{itemize}
        \item Clean, well-formatted, and understandable code.
        \item Consistent naming conventions and style.
        \item Logical organization (functions, classes, modules).
    \end{itemize}
    \vspace{0.5em}
    \item \textbf{Modularity and Reusability (15\%)}
    \begin{itemize}
        \item Code broken into manageable, reusable components.
        \item Avoidance of monolithic code blocks.
    \end{itemize}
}
\only{
    \item \textbf{Efficiency and Performance (10\%)}
    \begin{itemize}
        \item Consideration for efficiency of algorithms/data structures.
        \item Avoidance of obvious performance bottlenecks.
    \end{itemize}
    \vspace{0.5em}
    \item \textbf{Error Handling and Robustness (10\%)}
    \begin{itemize}
        \item Basic error handling for invalid inputs/conditions.
        \item Code is reasonably robust for typical usage.
    \end{itemize}
}
\end{itemize}
\end{frame}

%-------------------------------------------------------------------------------

\begin{frame}[t]
\frametitle{Code Review: Documentation and Comments (20\% of Code Review Grade)}
\begin{itemize}
    \item \textbf{Inline Comments (10\%)}
    \begin{itemize}
        \item Sufficient and meaningful comments for complex/non-obvious code.
        \item Comments aid in understanding the code's logic.
    \end{itemize}
    \vspace{0.5em}
    \item \textbf{Function/Module Documentation (10\%)}
    \begin{itemize}
        \item Clear docstrings or header comments for key functions/modules (purpose, parameters, returns).
        \item README or equivalent for setup and execution.
    \end{itemize}
\end{itemize}
\end{frame}

%-------------------------------------------------------------------------------

\begin{frame}[t]
\frametitle{Code Review: Correctness and Functionality (20\% of Code Review Grade)}
\begin{itemize}
    \item \textbf{Adherence to Project Specification (10\%)}
    \begin{itemize}
        \item Code correctly implements core requirements and chosen additional features of the Capstone Project.
    \end{itemize}
    \vspace{0.5em}
    \item \textbf{Logical Soundness (10\%)}
    \begin{itemize}
        \item Logic within the code is sound and correctly translates model rules.
        \item Absence of significant logical flaws or bugs.
    \end{itemize}
\end{itemize}
\end{frame}

%-------------------------------------------------------------------------------

\begin{frame}[t]
\frametitle{Code Review: Discussion and Response to Review (10\% of Code Review Grade)}
\small (Applicable if an interactive review session is held)
\begin{itemize}
    \item \textbf{Explanation of Choices (5\%)}
    \begin{itemize}
        \item Ability to clearly articulate design decisions and justify coding choices made during development.
    \end{itemize}
    \vspace{0.5em}
    \item \textbf{Responsiveness to Feedback (5\%)}
    \begin{itemize}
        \item Openness to constructive criticism.
        \item Ability to discuss potential improvements or alternatives thoughtfully.
    \end{itemize}
\end{itemize}
\end{frame}

\end{document}