\documentclass[a4paper,12pt]{article}
\usepackage[utf8]{inputenc}
\usepackage{amsmath, amssymb}
\usepackage{graphicx}
\usepackage{hyperref}
\usepackage{geometry}
\geometry{a4paper, margin=1in}
\usepackage{cite}

\title{Title of the Research Paper}
\author{Author Name}
\date{\today}

\begin{document}

\maketitle

\begin{abstract}
The abstract is your paper in a nutshell. It provides a brief summary of the research problem, methods, results, and conclusions. Typically, it should be around 150-250 words.
\end{abstract}

\section{Introduction}
The Introduction section provides background information, states the research problem, explains the significance of the study, and outlines the objectives. You may divide the introduction into five parts:
\begin{enumerate}
    \item \textbf{Hook}: The opening paragraph should often start with a sentence that grabs the reader's attention. This is called a hook.
    \item \textbf{A broad introduction to the field}: Start with a broad context before narrowing down to the specific issue. In the case of this report, you should identify and mention one or two important studies in the field related to your project.
    \item \textbf{The specific topic of study}: Quote papers closely related to your task and recent papers on the topic. The aim is to find at least two or three such papers.
    \item \textbf{Objectives of your work}: Clearly state what you are trying to do. Note that this might differ from a standard research paper, which usually aims to bring about new knowledge. If your goal was only to reproduce a result, say so.
    \item \textbf{Overview}: Conclude with a brief overview of the paper's structure.
\end{enumerate}

\section{Methods}
The Methods section describes how the research was conducted. It should provide enough detail to allow for replication by a colleague in your field. 

\begin{itemize}
    \item \textbf{Data Sources}: If you used external data sources (e.g., population of a country), describe the primary data source.
    \item \textbf{Materials or Tools}: Specify any software, models, or simulations used. Include the version numbers of software and any additional libraries or packages that were used.
    \item \textbf{Simulation or Modeling Procedure}: Outline the main steps of the simulation or modeling process but avoid including raw code. You may use text description, mathematical formulas, diagrams, or pseudocode. Instead of including full code, refer to it in an appendix or supplementary material if necessary.
    
    When using well-known models such as Logistic Growth or the Lotka-Volterra model, you should include the relevant equations and descriptions of variables and provide a brief explanation of the model.
    \item \textbf{Analysis}: If your paper includes data analysis, describe the methods of data analysis, including any statistical methods.
\end{itemize}

\section{Results}
The results section presents the findings of the study. Don't interpret your data here. Use tables, figures, and graphs to present data. Figure captions should be concise but comprehensive. They should describe the data shown and draw attention to important features contained within the figure.

\section{Discussion}
The discussion section interprets and explains the significance of the findings, relating the results to your objectives and the existing literature. Attempt at least two or three paragraphs. Begin with a summary of the key findings. Discuss any unexpected results and potential reasons for these. Address the limitations of the study. Discuss what the results imply for future research.

\section{Conclusion}
The conclusion section summarizes the main findings and their significance. You should restate your research objective and how it was addressed. Highlight the key findings and their implications.

\section{References}
The reference section lists all sources cited in the paper.
\bibliographystyle{plain}
\bibliography{references} % Create a file named 'references.bib' for your references.

\end{document}
